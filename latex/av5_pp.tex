\documentclass[a4paper]{exam}
%\usepackage{ucs}
\usepackage[T2A]{fontenc}
\usepackage[utf8]{inputenc}
\usepackage[english,bulgarian]{babel}
\usepackage{graphicx}
\usepackage{url}
\usepackage{textcomp}
\usepackage{amsmath}
\usepackage{amsfonts}
\usepackage{amssymb}
\usepackage{tabularx}
\usepackage{array}
\usepackage[margin=1.5in]{geometry}
\usepackage[unicode]{hyperref}

%\usepackage{fancyhdr}
\setlength{\headheight}{15pt}
 
%\pagestyle{fancyplain}
\pagestyle{headandfoot}
\firstpageheadrule
\runningheadrule

\usepackage{listings}
\lstset{language=Java,captionpos=b,
tabsize=4,frame=lines,
basicstyle=\small\ttfamily,
keywordstyle=\color{blue},
commentstyle=\color{gray},
stringstyle=\color{violet},
breaklines=true,showstringspaces=false}


\addto\captionsbulgarian{%
  \renewcommand{\contentsname}%
    {Содржина}%
  \renewcommand{\tablename}%
    {Табела}%
  \renewcommand{\figurename}%
    {Слика}%
  \renewcommand{\bibname}%
    {Библиографија}%
  \renewcommand{\listfigurename}%
    {Листа на слики}%
  \renewcommand{\listtablename}%
    {Листа на табели}%
}

\rhead{\textsc{Напредно програмирање}}
\chead{Задачи}
\lhead{Аудиториски вежби 5}
\lfoot{}
\cfoot{\thepage}
\rfoot{}
\usepackage{fancyvrb}
\usepackage{xcolor}
\usepackage{textcomp}

\begin{document}
\begin{questions}

%\section{}

\question
Да се напише програма во која се читаат името, количината и цената на три
производи. Името може да содржи празни места. Да се отпечати сметка со пресметка
на данок од 19\%. Сите цени треба да се печатат со две децимални места. Сметката
треба да биде форматирана во колони со 30 знаци за името, 10 знаци за
количината, 10 знаци за цената и 10 знаци за вкупно. 

\emph{Пример влез и излез:}

\begin{verbatim}
Name                            Quantity     Price     Total
lollipops                             10      0.50      5.00
diet soda                              3      1.25      3.75
chocolate bar                         20      0.75     15.00
Subtotal                                               23.75
6.25% sales tax                                         1.48
Total                                                  25.23
\end{verbatim}

\lstinputlisting{src/av5/ProductsBill.java}

\question
Ваша задача е да распределите три еднакви награди на 30 финалисти. За секој
финалист има назначено број од 1 до 30. Напишете програма која случајно ги
избира броевите на 3-те финалисти кои треба да добијат награда. Треба да
внимавате на некои ограничувања при избирањето на наградените. На пример,
одбирање на финалисти со броеви 3, 15 и 29 е валидно, но избирањето на 3, 3 и 31 како
добитници не е валидно, затоа што финалистот со број 3 е избран два пати, а 31
не е валиден број на финалист.

\lstinputlisting{src/av5/Finalists.java}

\question
Во датотеката \texttt{words.txt} има 99,171 зборови од Англискиот јазик. Да се
напише програма која ќе го пронајде најдолгиот збор палиндром.

\lstinputlisting{src/av5/LongestPalindrome.java}

\question
Задачата е дел од ``Niffty Assignment'' од авторот Steve Wolfman (http://nifty.
stanford.edu/2006/wolfman-pretid). Дадена ни е листа на броеви од податочни
извори од реалниот живот, на пример, листа со број на студенти запишани на
различни курсеви, бројот на коментари на различни Facebook статуси, бројот на
книги во различни библиотеки, бројот на гласови по избирачко место, итн. Логични
би било почетната цифра на секој број во листата да биде 1-9 со приближно 
еднаква веројатност. Меѓутоа, законот на Бенфорд (Benford’s Law) тврди дека
почетната цифра 1 е се појавува околу 30\% од времето и оваа вредност опаѓа со
големината на цифрата. Почетна цифра 9 се појавува само околу 5\% од времето.

Да се напише програма која го тестира законот на Бенфорд. Соберете листа од
најмалку 10 броеви од извори од реланиот живот и ставете ги во текстуална
датотека. Вашата програма треба ги измине сите броеви и треба да изброи колку
броеви се со прва цифра 1, колку со прва цифра 2, итн. За секоја цифра да се
отпечати процентот на застапеност како прва цифра.

\lstinputlisting{src/av5/BenfordLawTest.java}

\question
Следниот код е дизајниран од J. Hacker за видео игра. Постои класа \texttt{Alien} која
репрезентира вонземјанин и класа \texttt{AlienPack} која репрезентира група вонземјани и
колку штета може да нанесат:

\lstinputlisting{src/av5/AlienPack.java}

Кодот не е многу објектно ориентиран и не подржува криење на информациите во
класата \texttt{Alien}. Да се пренапише, така што ќе се искористи наследување за
да се репрезентираат различни типови вонземјани, наместо да се користи
параметарот \texttt{``type''}. Исто така пренапишете ја класата \texttt{Alien}
така што ќе ги крие инстанцните променливи и креирајте метод \texttt{getDamage}
кој за секоја од изведените класа ќе ја враќа штетата која ја предизвикува. На
крај пренапишете го методот \texttt{calculateDamage} да го користи
\texttt{getDamage} и напишете \texttt{main} метод да ја тестирате класата.

\question
Креирајте класа Movie. Во класата се чуваат рејтинг (MPAA рејтинг, пр. Rated G,
PG-13, R), број за идентификација (ID) и име на филмот со соодветни методи за
менување/пристап. Исто така преоптоварете го методот \texttt{equals()} во кој
два филма се споредуваат според нивниот ID. Следно, креирајте дополнителни три
класи \texttt{Action}, \texttt{Comedy} и \texttt{Drama} кои наследуваат од
\texttt{Movie}. На крај, преоптоварен метод \texttt{calcLateFees} кој пресметува
доплата за доцнење при враќање. Основната доплата е \$2/ден. Акционите филмови
имаат доплата \$3/ден, комедиите \$2.50/ден, а драмите \$2/ден.

\end{questions}

\end{document}
