\documentclass[a4paper]{exam}
%\usepackage{ucs}
\usepackage[T2A]{fontenc}
\usepackage[utf8]{inputenc}
\usepackage[english,bulgarian]{babel}
\usepackage{graphicx}
\usepackage{url}
\usepackage{textcomp}
\usepackage{amsmath}
\usepackage{amsfonts}
\usepackage{amssymb}
\usepackage{tabularx}
\usepackage{array}
\usepackage[margin=1.5in]{geometry}
\usepackage[unicode]{hyperref}

%\usepackage{fancyhdr}
\setlength{\headheight}{15pt}
 
%\pagestyle{fancyplain}
\pagestyle{headandfoot}
\firstpageheadrule
\runningheadrule

\usepackage{listings}
\lstset{language=Java,captionpos=b,
tabsize=4,frame=lines,
basicstyle=\small\ttfamily,
keywordstyle=\color{blue},
commentstyle=\color{gray},
stringstyle=\color{violet},
breaklines=true,showstringspaces=false}


\addto\captionsbulgarian{%
  \renewcommand{\contentsname}%
    {Содржина}%
  \renewcommand{\tablename}%
    {Табела}%
  \renewcommand{\figurename}%
    {Слика}%
  \renewcommand{\bibname}%
    {Библиографија}%
  \renewcommand{\listfigurename}%
    {Листа на слики}%
  \renewcommand{\listtablename}%
    {Листа на табели}%
}

\rhead{\textsc{Напредно програмирање}}
\chead{Задачи}
\lhead{Аудиториски вежби 4}
\lfoot{}
\cfoot{\thepage}
\rfoot{}
\usepackage{fancyvrb}
\usepackage{xcolor}
\usepackage{textcomp}

\begin{document}
\begin{questions}

\section{\texttt{ArrayList}}

\question
Во спортот скокови во вода, седум судии го оценуваат скокот со оценка од 0 до
10, со што оценката може да биде децимален број. Најдобрата и најлошата оценка
се исфрлаат, а останатите се собираат. Сумата потоа се множи со степенот на
тежина на тој скок. Степенот на тежина е број од 1.2 до 3.8 поени. Вкупниот збир
потоа се множи со 0.6 и се добива крајниот резултат на скокачот. Да се напише
компјутерска програма која за дадени степен на тежина и оценките од
седумте судии, ќе го пресмета резултатот од скокот. Програмата треба да користи
\texttt{ArrayList} со тип \texttt{Double} за оценките од судиите.

\lstinputlisting{src/av4/Dive.java}

\question

Со помош на Глобалниот Систем за Позиционирање (GPS) може да добиваме патеки на
движење. Патека на движење се состои од коориданите на локации на мапа заедно со
соодветни времиња (timestamp). Нашиот GPS систем запишува точки на движење кои
се состојат од (X, Y) координати на мапа заедно со времиња (timestamp t) во кои
се запишани бројот на секунди кои поминале откако е вклучен уредот. Да се напише
програма во која од многу точки кои се читаат и се сместуваат во
\texttt{ArrayList}, со помош на класа за репрезетнација на точка. Секоја точка е
последователна локација од движењето по некоја планинарска рута. Координатите се
од тип \texttt{double}, а времето од тип \texttt{integer}. Вашата програма треба
да го пресметува вкупното поминато растојание и просечната брзина во километри
на час. Мапата е скалирана со фактор 1 = 0.1 километар. На пример, ако две точки
се на координати (X = 1, Y = 1, T = 0) и (X = 2, Y = 1, T = 3600), тогаш
вкупното поминато растојание е 0.1 километар за 3600 секунди, односно со брзина
0.1 km/h.

\lstinputlisting{src/av4/Waypoint.java}
\lstinputlisting{src/av4/HikerTrack.java}

\section{Генеричко програмирање}

\question
Да се напише генеричка класа кој симулира исцртување на случаен предмет од
некоја кутија. Оваа класа треба да се користи за случајно исцртување. На пример,
класата може да содржи листа со имиња и избира едно случајно име, или пак листа
со броеви за лотарија и избира случајно број. Креирајте метод \texttt{add} за
додавање објект од соодветниот тип и метод \texttt{isEmpty} кој проверува дали
кутијата е празна. На крај, имплементирајте метод \texttt{drawItem} кој случајно
избира објект од кутијата и го враќа назад. Ако се обидеме да цртаме со празна
кутија се враќа \texttt{null}. Да се напише \texttt{main} метод кој ја тестира
класата.
\lstinputlisting{src/av4/Box.java}

\question
Да се имплементира класа за податочна структура \texttt{PriorityQueue} со помош
на \texttt{ArrayList}. \texttt{PriorityQueue} е податочна структура во која
секоја елемент се додава заедно со неговиот приоритет (цел број). Приоритет да
се дефинира така што оние елементи со најголема вредност на приоритет имаат
повисок приоритет. Класата треба да ги имплементира следните методи:

\begin{itemize}
  \item \texttt{add(item, priority)} - Додава нов елемент со асоциран
  приоритет.
\item \texttt{remove()} - Го враќа елементот со најголем приоритет и го
брише од редот. Ако редот е празен се враќа \texttt{null}.
\end{itemize}

Пример за \texttt{priority queue} со стрингови: 
\begin{verbatim}
q.add("X", 10);
q.add("Y", 1);
q.add("Z", 3);
System.out.println(q.remove()); // Returns X
System.out.println(q.remove()); // Returns Z
System.out.println(q.remove()); // Returns Y
\end{verbatim}

Тестирајте го редот со податоци со приоритет во различен редослед (пр.,
растечки, опаѓачки, мешан). Редот може да се имплементира со линеарно
пребарување низ \texttt{ArrayList}. 

\lstinputlisting{src/av4/PriorityQueue.java}

\question
Да се напише класа \texttt{MyMathClass}, во која ќе се имплементира статички
метод \texttt{standardDeviation} кој како аргумент прима \texttt{ArrayList} од
тип Т, каде што T е нумерички тип (пр. Integer, Double, или било која класа која
наследува од \texttt{java.lang.Number}) и враќа резултат \texttt{double} кој
претставува стандардната девијација на вредностите во листата. Стандардна
девијација се пресметува со следната формула: $\sigma =
  \sqrt{\frac{1}{N} \sum_{i=1}^N (x_i - \mu)^2}, \mu =
  \frac{1}{N} \sum_{i=1}^N x_i$

Вашата програма треба да генерира грешка при компајлирање ако методот за
пресметување стандардна девијација се повика со \texttt{ArrayList} која е
дефинирана со ненумерички тип (пр. \texttt{String}).

\lstinputlisting{src/av4/MyMathClass.java}

\end{questions}

\end{document}
