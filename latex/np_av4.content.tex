\section{Java пакети}

\begin{frame}\frametitle{Класата \texttt{ArrayList}}
\begin{itemize}
\item
  \texttt{ArrayList} е класа од стандарадните библиотеки во Java
  \item За разлика од низите, кои имаат фиксна должина откако ќе се креираат,
  \texttt{ArrayList} е објект кој може да се проширува додека програмата се
  извршува
	\item Генерално, \texttt{ArrayList} ја има истата улога како и низите, со тоа
	што \texttt{ArrayList} може да ја менува својата должина додека програмата се
	извршува
	\item \texttt{ArrayList} е имплементирана со помош на низа како приватна
	инстацна променлива
\end{itemize}
\end{frame}

\begin{frame}\frametitle{Зошто секогаш да не користиме \texttt{ArrayList}
наместо низи?}

\begin{itemize}
\item
  \texttt{ArrayList} е по неефикасна од низа
  \item Не подржува нотација на големи загради
  \item Основниот тип на \texttt{ArrayList} мора да биде класа (или друг тип
  референца): не може да биде примитивен тип
\end{itemize}
\end{frame}

\begin{frame}[fragile]{Користење на \texttt{ArrayList}}
\begin{itemize}
  \item За да се користи \texttt{ArrayList}, треба најпрво да се вклучи пакетот
  \texttt{java.util}
  \item \texttt{ArrayList} се креира и именува на ист начин како и објект од
  било која класа, освен што мора да се специфицира нејзиниот тип на следниов
  начин:
  \begin{lstlisting}
  ArrayList<BaseType> aList = new ArrayList<BaseType>();
  \end{lstlisting}
  \item Почетниот капацитет може да се проследи како аргумент на конструкторот
  \item Следниот код креира \texttt{ArrayList} кој чува објекти од тип
  \texttt{String} и има почетен капацитет  од 20 членови
  \begin{lstlisting}
  ArrayList<String> list = new ArrayList<String>(20);
  \end{lstlisting} 
  \item Специфицирање на почетниот капацитет не ја ограничува
  големината до која \texttt{ArrayList} може да расте
\end{itemize}
\end{frame}

\begin{frame}[fragile]{Користење на \texttt{ArrayList}}
\begin{itemize}
  \item Методот \texttt{add} се користи за дадовање на елемент во
  \texttt{ArrayList}
  \item Методот \texttt{add} е преоптоварен
  \item Постои и верзија со два аргументи кој овозможува да се додаде
  елемент на одредена позиција зададена преку индекс
  \item Методот \texttt{size} се користи да се открие бројот на елементи во \texttt{ArrayList}
  \begin{lstlisting}
  int howMany = list.size();
  \end{lstlisting}
  \item Методот \texttt{set} се користи за менување на постоечки елемент, додека
  методот \texttt{get} се користи за пристапување до одреден постоечки елемент.
  И двата методи го примаат за аргумент индексот (0 базиран) на елементот кој
  сакаме да го пристапиме.
  \begin{lstlisting}
  list.set(index, "something else");
  String thing = list.get(index);
  \end{lstlisting}
\end{itemize}
\end{frame}


