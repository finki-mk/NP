\documentclass[a4paper]{exam}
%\usepackage{ucs}
\usepackage[T2A]{fontenc}
\usepackage[utf8]{inputenc}
\usepackage[english,bulgarian]{babel}
\usepackage{graphicx}
\usepackage{url}
\usepackage{textcomp}
\usepackage{amsmath}
\usepackage{amsfonts}
\usepackage{amssymb}
\usepackage{tabularx}
\usepackage{array}
\usepackage[margin=1.5in]{geometry}
\usepackage[unicode]{hyperref}

%\usepackage{fancyhdr}
\setlength{\headheight}{15pt}
 
%\pagestyle{fancyplain}
\pagestyle{headandfoot}
\firstpageheadrule
\runningheadrule

\usepackage{listings}
\lstset{language=Java,captionpos=b,
tabsize=4,frame=lines,
basicstyle=\small\ttfamily,
keywordstyle=\color{blue},
commentstyle=\color{gray},
stringstyle=\color{violet},
breaklines=true,showstringspaces=false}


\addto\captionsbulgarian{%
  \renewcommand{\contentsname}%
    {Содржина}%
  \renewcommand{\tablename}%
    {Табела}%
  \renewcommand{\figurename}%
    {Слика}%
  \renewcommand{\bibname}%
    {Библиографија}%
  \renewcommand{\listfigurename}%
    {Листа на слики}%
  \renewcommand{\listtablename}%
    {Листа на табели}%
}

\rhead{\textsc{Напредно програмирање}}
\chead{Задачи}
\lhead{Аудиториски вежби 1}
\lfoot{}
\cfoot{\thepage}
\rfoot{}
\usepackage{fancyvrb}
\usepackage{xcolor}
\usepackage{textcomp}

\begin{document}
\begin{questions}

\question

Да се напише програма која ќе ги најде сите парови позитивни цели броеви (a, b)
такви што a < b < 1000 и $\frac{(a^2 + b^2 + 1)}{a * b}$ е цел број.

решение:

\lstinputlisting{src/av1/Ex1.java}

\question

Да се напише метод кој ќе прима еден цел број и ќе ја печати неговата
репрезентација како Римски број. Пример. ако ако се повика со парамететар 1998,
излезот треба да биде MCMXCVIII.

\textbf{за дома}

\question
Ваша задача е да печатите броеви во средни загради, форматирани на следниот
начин: [1][2][3], итн. Напишете метод кој прима два параметри: howMany и
lineLength и ги печати броевите од 1 до howMany во претходно опишаниот формат,
со што не смее да се печатат повеќе знаци во една линија од lineLength. Не треба
да се започне со отворена заграда [ ако не може да се затвори во истата линија
со соодветна ].

\textbf{за дома}

\end{questions}

\end{document}
